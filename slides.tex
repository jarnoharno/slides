\documentclass{beamer}
\usepackage[utf8]{inputenc}
\usepackage{amsmath}
\usepackage{mathtools}
\usepackage{bm}
%\usetheme{Berkeley}
\usecolortheme{beaver}

\setbeamercovered{transparent}

\begin{document}

\title{The pinhole camera}
\subtitle{Solving the three geometric problems}
\author{Jarno Leppänen}
\date{13.10.2015}
\frame{\titlepage}

\section{Recap}

\begin{frame}
  \frametitle{Pinhole Camera Model}
  \begin{align*}
    \mathbf{pinhole}[\mathbf{w},\boldsymbol{\Lambda},\boldsymbol{\Omega},
    \bm{\tau}] &=
      \begin{bmatrix*}[l]
        \frac{\phi_x (\omega_{11}u+\omega_{12}v+\omega_{13}w+\tau_x)
          +\gamma(\omega_{21}u+\omega_{22}v+\omega_{23}w+\tau_y)
        }{\omega_{31}u+\omega_{32}v+\omega_{33}w+\tau_z}+\delta_x \\
        \frac{\phi_y (\omega_{21}u+\omega_{22}v+\omega_{23}w+\tau_y)
        }{\omega_{31}u+\omega_{32}v+\omega_{33}w+\tau_z}+\delta_y \\
      \end{bmatrix} \\
      \\
      \uncover<2->{
    \Pr(\mathbf{x}|\mathbf{w},\boldsymbol{\Lambda},\boldsymbol{\Omega},
    \bm{\tau}) &=
    \operatorname{Norm}_\mathbf{x}[\mathbf{pinhole}[\mathbf{w},
    \boldsymbol{\Lambda},\boldsymbol{\Omega},\bm{\tau}],\sigma^2\mathbf{I}]
  }
  \end{align*}
  \uncover<3->{
  \begin{itemize}
    \item $\boldsymbol{\Lambda}$ are the \emph{intrinsic parameters}
    \item $\boldsymbol{\Omega},\bm{\tau}$ are the \emph{extrinsic parameters}
  \end{itemize}
}
\end{frame}

\begin{frame}
  \frametitle{Learning Extrinsic Parameters}
  \includegraphics[width=1.0\textwidth]{fig/recap1.png}
  \begin{align*}
    \boldsymbol{\hat{\Omega}},\bm{\hat{\tau}} &=
    \underset{\boldsymbol{\Omega},\bm{\tau}}
    {\operatorname{argmax}} \left[
      \sum_{i=1}^I \log{\left[
    \Pr(\mathbf{x}_i|\mathbf{w}_i,\boldsymbol{\Lambda},\boldsymbol{\Omega},
    \bm{\tau})
      \right]}
    \right]
  \end{align*}
  \begin{itemize}
    \item<2-> $I$ is the number of points
    \item<3-> Position and orientation of the camera
  \end{itemize}
\end{frame}

\begin{frame}
  \frametitle{Learning Intrinsic Parameters}
  \includegraphics[width=1.0\textwidth]{fig/recap2.png}
  \begin{align*}
    \boldsymbol{\hat{\Lambda}} &=
    \underset{\boldsymbol{\Lambda}}
    {\operatorname{argmax}} \left[
    \underset{\boldsymbol{\Omega},\bm{\tau}}
    {\operatorname{max}} \left[
      \sum_{i=1}^I \log{\left[
    \Pr(\mathbf{x}_i|\mathbf{w}_i,\boldsymbol{\Lambda},\boldsymbol{\Omega},
    \bm{\tau})
      \right]}
  \right]\right]
  \end{align*}
  \begin{itemize}
    \item<2-> $I$ is the number of points
    \item<3-> Focal length and skew calibration
  \end{itemize}
\end{frame}

\begin{frame}
  \frametitle{Inferring 3D World Points}
  \includegraphics[width=1.0\textwidth]{fig/recap3.png}
  \begin{align*}
    \mathbf{\hat{w}} &=
    \underset{\mathbf{w}}
    {\operatorname{argmax}} \left[
      \sum_{j=1}^J \log{\left[
    \Pr(\mathbf{x}_j|\mathbf{w},\boldsymbol{\Lambda}_j,\boldsymbol{\Omega}_j,
    \bm{\tau}_j)
      \right]}
  \right]
  \end{align*}
  \begin{itemize}
    \item<2-> $J \geq 2$ is the number of cameras
    \item<3-> 3D object reconstruction
  \end{itemize}
\end{frame}

\begin{frame}
  \frametitle{Solving the Problems}
  \begin{itemize}[<+->]
    \item Each objective function is nonconvex
    \item No closed form optimal solutions
    \item Need nonlinear optimization
    \item Need good initial estimates to ensure convergence to global maximum
  \end{itemize}
\end{frame}

\begin{frame}
  \frametitle{Strategy}
  \begin{itemize}[<+->]
    \item Choose new easier objective functions with closed form solutions
    \item Use output as an initial guess to the nonlinear optimization procedure
    \item Find global optimum to the true objective function
  \end{itemize}
\end{frame}

\section{Homogeneous Coordinates}

\begin{frame}
  \frametitle{Homogeneous Coordinates}
  \begin{itemize}[<+->]
    \item Change the representation of 2D and 3D points
    \item Projection equations become linear with the new representation
    \item Resulting modified objective functions minimize a different
      error measure
    \item Solutions are \emph{not} guaranteed to be the same as those for the
      original problem
  \end{itemize}
\end{frame}

\begin{frame}
  \frametitle{Definition}
  \begin{itemize}[<+->]
    \item 2D homogeneous coordinate: $
    \mathbf{\tilde{x}} =
    \begin{bmatrix}
      \tilde{x} \\ \tilde{y} \\ \tilde{z}
    \end{bmatrix}
      = \lambda
    \begin{bmatrix}
      x \\ y \\ 1
    \end{bmatrix}$
  \item Redundant, e.g. $\mathbf{\tilde{x}} = [2,4,2]^T$ and
  $\mathbf{\tilde{x}} = [3,6,3]^T$ both represent the 2D image point
  $\mathbf{x} = [1,2]^T$
\item Conversion to 2D image point: $\mathbf{x} = \begin{bmatrix}
    \tilde{x}/\tilde{z} \\ \tilde{y}/\tilde{z}
  \end{bmatrix}
  $
\item 3D points analogously: $
    \mathbf{\tilde{w}} =
      \lambda
    \begin{bmatrix}
      u \\ v \\ w \\ 1
    \end{bmatrix}$
\end{itemize}
\end{frame}

\begin{frame}
  \frametitle{Geometric Interpretation}
  \begin{columns}
    \column{.5\textwidth}
    \begin{itemize}[<+->]
      \item Homogeneous coordinate defines a ray through the origin
      \item Can represent points at infinity
    \end{itemize}
    \column{.5\textwidth}
      \includegraphics[width=1.0\textwidth]{fig/homogeneous.png}
    \end{columns}
\end{frame}

\begin{frame}
  \frametitle{Camera Model in Homogeneous Coordinates}
  \begin{itemize}[<+->]
    \item Pinhole projection: $\mathbf{x} = \begin{bmatrix}
        \frac{\phi_x u + \gamma v}{w}+\delta_x \\
        \frac{\phi_y v}{w}+\delta_y
      \end{bmatrix}$
    \item In homogeneous coordinates: \begin{equation*}
      \lambda_1 \begin{bmatrix}
        x \\ y \\ 1
      \end{bmatrix} =
      \lambda_2
      \begin{bmatrix}
        \phi_x & \gamma & \delta_x & 0 \\
        0 & \phi_y & \delta_y & 0 \\
        0 & 0 & 1 & 0
      \end{bmatrix} \begin{bmatrix}
        u \\ v \\ w \\ 1
      \end{bmatrix}
      \end{equation*}
    \item Indeed:
      \begin{align*}
        \lambda x &= \phi_x u + \gamma v + \delta_x w \\
        \lambda y &= \phi_y v + \delta_y w \\
        \lambda &= \lambda_1 / \lambda_2 = w
      \end{align*}
  \end{itemize}
\end{frame}

\begin{frame}
  \frametitle{Complete Model}
    \begin{align*}
      \lambda \begin{bmatrix}
        x \\ y \\ 1
      \end{bmatrix} =
      \begin{bmatrix}
        \phi_x & \gamma & \delta_x & 0 \\
        0 & \phi_y & \delta_y & 0 \\
        0 & 0 & 1 & 0
      \end{bmatrix}
      \begin{bmatrix}
        \omega_{11} & \omega_{12} & \omega_{13} & \tau_x \\
        \omega_{21} & \omega_{22} & \omega_{23} & \tau_y \\
        \omega_{31} & \omega_{32} & \omega_{33} & \tau_z \\
        0 & 0 & 0 & 1
      \end{bmatrix} \begin{bmatrix}
        u \\ v \\ w \\ 1
      \end{bmatrix}
      \end{align*}
      \pause
      or
      \begin{align*}
        \mathbf{\tilde{x}} &= \boldsymbol{\Lambda}
        \begin{bmatrix}
          \boldsymbol{\Omega} &
          \bm{\tau}
        \end{bmatrix}
          \mathbf{\tilde{w}}
      \end{align*}
      \pause
      \begin{itemize}
        \item This model will be used to compute good initial estimates
          for the nonlinear optimization problem
      \end{itemize}
\end{frame}

\section{Learning extrinsic parameters}

\begin{frame}
  \frametitle{Learning extrinsic parameters}
  \begin{align*}
    \boldsymbol{\hat{\Omega}},\bm{\hat{\tau}} &=
    \underset{\boldsymbol{\Omega},\bm{\tau}}
    {\operatorname{argmax}} \left[
      \sum_{i=1}^I \log{\left[
    \Pr(\mathbf{x}_i|\mathbf{w}_i,\boldsymbol{\Lambda},\boldsymbol{\Omega},
    \bm{\tau})
      \right]}
    \right]
  \end{align*}
  \pause
  \begin{itemize}
    \item For the $i$th point in homogeneous coordinates:
    \begin{align*}
      \lambda_i \begin{bmatrix}
        x_i \\ y_i \\ 1
      \end{bmatrix} =
      \begin{bmatrix}
        \phi_x & \gamma & \delta_x & 0 \\
        0 & \phi_y & \delta_y & 0 \\
        0 & 0 & 1 & 0
      \end{bmatrix}
      \begin{bmatrix}
        \omega_{11} & \omega_{12} & \omega_{13} & \tau_x \\
        \omega_{21} & \omega_{22} & \omega_{23} & \tau_y \\
        \omega_{31} & \omega_{32} & \omega_{33} & \tau_z \\
        0 & 0 & 0 & 1
      \end{bmatrix} \begin{bmatrix}
        u_i \\ v_i \\ w_i \\ 1
      \end{bmatrix}
      \end{align*}
      \pause
    \item We want to eliminate known $\boldsymbol{\Lambda}$
  \end{itemize}
\end{frame}

\begin{frame}
  \frametitle{Learning extrinsic parameters}
  \begin{itemize}[<+->]
    \item Multiply by $\boldsymbol{\Lambda}^{-1}$:
    \begin{align*}
      \lambda_i \begin{bmatrix}
        x'_i \\ y'_i \\ 1
      \end{bmatrix} =
      \begin{bmatrix}
        \omega_{11} & \omega_{12} & \omega_{13} & \tau_x \\
        \omega_{21} & \omega_{22} & \omega_{23} & \tau_y \\
        \omega_{31} & \omega_{32} & \omega_{33} & \tau_z \\
        0 & 0 & 0 & 1
      \end{bmatrix} \begin{bmatrix}
        u_i \\ v_i \\ w_i \\ 1
      \end{bmatrix}
      \end{align*}
    \item Here $\mathbf{\tilde{x}'} =
      \boldsymbol{\Lambda}^{-1}\mathbf{\tilde{x}}$ are \emph{normalized image
      coordinates}
    \item Now $\lambda_i = \omega_{31}u_i+\omega_{32}v_i+\omega_{33}w_i+\tau_z$

  \end{itemize}
\end{frame}

\setcounter{MaxMatrixCols}{20}

\begin{frame}
  \frametitle{Learning extrinsic parameters}
  \begin{itemize}[<+->]
        \item Substituting back:
          \begin{align*}
          \begin{bmatrix}
            (\omega_{31}u_i+\omega_{32}v_i+\omega_{33}w_i+\tau_z)x'_i \\
            (\omega_{31}u_i+\omega_{32}v_i+\omega_{33}w_i+\tau_z)y'_i
          \end{bmatrix} &=
            \begin{bmatrix}
              \omega_{11} & \omega_{12} & \omega_{13} & \tau_x \\
              \omega_{21} & \omega_{22} & \omega_{23} & \tau_y
            \end{bmatrix}
            \begin{bmatrix}
              u_i \\ v_i \\ w_i \\ 1
            \end{bmatrix}
        \end{align*}
      \item Rearranging for all $i$:
        \tiny{
        \begin{align*}
          \underbrace{
          \begin{bmatrix}
            u_1 & v_1 & w_1 & 1 & 0 & 0 & 0 & 0 &
            -u_1 x_1' & -v_1 x_1' & -w_1 x_1' & -x_1' \\
            0 & 0 & 0 & 0 & u_1 & v_1 & w_1 & 1 &
            -u_1 y_1' & -v_1 y_1' & -w_1 y_1' & -y_1' \\
            u_2 & v_2 & w_2 & 1 & 0 & 0 & 0 & 0 &
            -u_2 x_2' & -v_2 x_2' & -w_2 x_2' & -x_2' \\
            0 & 0 & 0 & 0 & u_2 & v_2 & w_2 & 1 &
            -u_2 y_2' & -v_2 y_2' & -w_2 y_2' & -y_2' \\
                      & & & & & & & \vdots & & & \\
            u_I & v_I & w_I & 1 & 0 & 0 & 0 & 0 &
            -u_I x_I' & -v_I x_I' & -w_I x_I' & -x_I' \\
            0 & 0 & 0 & 0 & u_I & v_I & w_I & 1 &
            -u_I y_I' & -v_I y_I' & -w_I y_I' & -y_I'
          \end{bmatrix}
        }_{
          \mathbf{A}
        }
        \underbrace{
          \begin{bmatrix}
            \omega_{11} \\
            \omega_{12} \\
            \omega_{13} \\
            \tau_x \\
            \omega_{21} \\
            \omega_{22} \\
            \omega_{23} \\
            \tau_y \\
            \omega_{31} \\
            \omega_{32} \\
            \omega_{33} \\
            \tau_z \\
          \end{bmatrix}
        }_{
          \mathbf{b}
        }
          &=
          \mathbf{0}
        \end{align*}
      }
      \end{itemize}
\end{frame}

\newcommand{\norm}[1]{\left\lVert#1\right\rVert}

\begin{frame}
  \frametitle{Computing estimates}
  \begin{itemize}[<+->]
    \item $\mathbf{A} \mathbf{b} = \mathbf{0}$ overdetermined
    \item \emph{Minimum direction problem}:
      $\operatorname{argmin}_\mathbf{b}\norm{\mathbf{Ab}}^2$ subject to
      $\norm{\mathbf{b}} = 1$
    \item Compute singular value decomposition $\mathbf{A} = \mathbf{ULV}^T$
      and set $\mathbf{\tilde{b}}$ to be the last column of $\mathbf{V}$
  \end{itemize}
\end{frame}

\begin{frame}
  \frametitle{Satisfying constraints}
  \begin{itemize}[<+->]
    \item Estimates of $\boldsymbol{\Omega}$ and $\bm{\tau}$ extracted from
      $\mathbf{b}$ have arbitrary scale
    \item $\boldsymbol{\Omega}$ must be a rotation matrix: find "closest" matrix
      to our estimate that satisfies constraints
    \item \emph{Orthogonal Procrustes problem}:
      $\operatorname{argmin}_{\boldsymbol{\Omega}}
      \norm{\boldsymbol{\Omega}-\mathbf{B}}$ subject to
      $\boldsymbol{\Omega\Omega}^T = \mathbf{I}$
    \item Compute SVD $\boldsymbol{\Omega} = \mathbf{ULV}^T$ and set
      $\boldsymbol{\hat{\Omega}} = \mathbf{UV}^T$
    \item Now $\bm{\tau}$ must be scaled:
      $\bm{\hat{\tau}} = \sum_{m=1}^3 \sum_{n=1}^3 \frac{\hat{\Omega}_{mn}}
      {\Omega_{mn}}\bm{\tau}$
    \item Finally, if $\tau_z < 0$ (object behind camera),
      multiply estimates by minus one
  \end{itemize}
\end{frame}

\begin{frame}
  \frametitle{Discussion}
  \begin{itemize}[<+->]
    \item Very ad hoc algorithm, typical of methods using homogeneous
      coordinates
    \item Can be inaccurate but usually sufficient for an initial guess
    \item Subsequent nonlinear optimization procedure must ensure that
      $\boldsymbol{\Omega}$ remains a valid rotation matrix
    \item The procedure requires at least 6 points for a unique solution (12
      unknowns, 2 equations per point)
  \end{itemize}
\end{frame}

\section{Learning intrinsic parameters}

\begin{frame}
  \frametitle{Learning intrinsic parameters}
  \begin{align*}
    \boldsymbol{\hat{\Lambda}} &=
    \underset{\boldsymbol{\Lambda}}
    {\operatorname{argmax}} \left[
    \underset{\boldsymbol{\Omega},\bm{\tau}}
    {\operatorname{max}} \left[
      \sum_{i=1}^I \log{\left[
    \Pr(\mathbf{x}_i|\mathbf{w}_i,\boldsymbol{\Lambda},\boldsymbol{\Omega},
    \bm{\tau})
      \right]}
  \right]\right]
  \end{align*}
  \begin{itemize}
    \item<2-> Coordinate ascent: alternately (1) optimize
      $\boldsymbol{\Omega}, \bm{\tau}$ for fixed $\boldsymbol{\Lambda}$ and
      (2) optimize $\boldsymbol{\Lambda}$ for fixed
      $\boldsymbol{\Omega}, \bm{\tau}$
    \item<3-> Converges slowly
    \item<4-> We already have a solution for the first step
    \item<5-> Second step has an easy solution
  \end{itemize}
\end{frame}

\begin{frame}
  \frametitle{Estimating $\boldsymbol{\Lambda}$}
  \begin{align*}
    \boldsymbol{\hat{\Lambda}}
    &=
    \underset{\boldsymbol{\Lambda}}
    {\operatorname{argmax}} \left[
      \sum_{i=1}^I \log{\left[
    \Pr(\mathbf{x}_i|\mathbf{w}_i,\boldsymbol{\Lambda},\boldsymbol{\Omega},
    \bm{\tau})
      \right]}
  \right]
\\
&=
    \underset{\boldsymbol{\Lambda}}
    {\operatorname{argmax}} \left[
      \sum_{i=1}^I \log{\left[
          \operatorname{Norm}_{\mathbf{x}_i}
          \left[
            \mathbf{pinhole}
            \left[
              \mathbf{w}_i,\boldsymbol{\Lambda},\boldsymbol{\Omega},\bm{\tau}
            \right]
            ,\sigma^2\mathbf{I}
          \right]
      \right]}
  \right]
\\
&=
    \underset{\boldsymbol{\Lambda}}
    {\operatorname{argmin}} \left[
      \sum_{i=1}^I
      \norm{
        \mathbf{x}_i-
            \mathbf{pinhole}
            \left[
              \mathbf{w}_i,\boldsymbol{\Lambda},\boldsymbol{\Omega},\bm{\tau}
            \right]
          }^2
  \right]
\\
  \end{align*}
  \begin{itemize}
    \item<2-> Least squares problem
  \end{itemize}
\end{frame}

\begin{frame}
  \begin{itemize}[<+->]
    \item $\mathbf{pinhole}[\ldots]$ is linear with respect to
      $\boldsymbol{\Lambda}$
    \item Can be written as $\mathbf{A}_i\mathbf{h}$ where
      $\mathbf{h}=[\phi_x,\gamma,\delta_x,\phi_y,\delta_y]^T$ and
  \end{itemize}
  \uncover<2->{
    {\tiny
  \begin{align*}
    \mathbf{A}_i &=
    \begin{bmatrix}
      \frac{\omega_{11}u_i+\omega_{12}v_i+\omega_{13}w_i+\tau_x}
      {\omega_{31}u_i+\omega_{32}v_i+\omega_{33}w_i+\tau_z} &
      \frac{\omega_{21}u_i+\omega_{22}v_i+\omega_{23}w_i+\tau_x}
      {\omega_{31}u_i+\omega_{32}v_i+\omega_{33}w_i+\tau_z} &
      1 & 0 & 0 \\
      0 & 0 & 0 &
      \frac{\omega_{21}u_i+\omega_{22}v_i+\omega_{23}w_i+\tau_y}
      {\omega_{31}u_i+\omega_{32}v_i+\omega_{33}w_i+\tau_z} &
      1
    \end{bmatrix}
  \end{align*}
  }
}
  \begin{itemize}
    \item<3-> Consequently, problem has the form
      \begin{align*}
        \mathbf{\hat{h}} &=
        \underset{\mathbf{h}}
    {\operatorname{argmin}} \left[
      \sum_{i=1}^I
      \left(\mathbf{A}_i\mathbf{h}-\mathbf{x}_i
      \right)^T
      \left(\mathbf{A}_i\mathbf{h}-\mathbf{x}_i
      \right)
  \right]
      \end{align*}
    \item<4-> Closed form solution
  \end{itemize}
\end{frame}

\begin{frame}
  \frametitle{Discussion}
  \begin{itemize}[<+->]
    \item Slow cenvergence
    \item In practice, perform a few coordinate ascent iterations and
        then use nonlinear optimization to optimize all parameters          simultaneously
  \end{itemize}
\end{frame}

\section{Inferring 3D world points}

\begin{frame}
  \frametitle{Inferring 3D world points}
  \begin{align*}
    \mathbf{\hat{w}} &=
    \underset{\mathbf{w}}
    {\operatorname{argmax}} \left[
      \sum_{j=1}^J \log{\left[
    \Pr(\mathbf{x}_j|\mathbf{w},\boldsymbol{\Lambda}_j,\boldsymbol{\Omega}_j,
    \bm{\tau}_j)
      \right]}
  \right]
  \end{align*}
  \begin{itemize}
    \item<2-> Need homogeneous coordinates
  \end{itemize}
\end{frame}

\begin{frame}
  \frametitle{Inferring 3D world points}
  \begin{itemize}[<+->]
    \item Using the same trick as before, for homogeneous world point
      $\mathbf{\tilde{w}}$ and the $j$th corresponding homogeneous image point
      $\mathbf{\tilde{x}}_j$ we have
      {\small
          \begin{align*}
          \begin{bmatrix}
            (\omega_{31}_j u+\omega_{32}_j v+\omega_{33}_j w+\tau_{zj})x'_j \\
            (\omega_{31}_j u+\omega_{32}_j v+\omega_{33}_j w+\tau_{zj})y'_j
          \end{bmatrix} &=
            \begin{bmatrix}
              \omega_{11j} & \omega_{12j} & \omega_{13j} & \tau_{xj} \\
              \omega_{21j} & \omega_{22j} & \omega_{23j} & \tau_{yj}
            \end{bmatrix}
            \begin{bmatrix}
              u \\ v \\ w \\ 1
            \end{bmatrix}
        \end{align*}
      }
    \item Here $x_j',y_j'$ are normalized image coordinates for image plane
      $j$
      \item Rearranging, we get
        {\small
        \begin{align*}
          \begin{bmatrix}
            \omega_{31j}x_j'-\omega_{11j} &
            \omega_{32j}x_j'-\omega_{12j} &
            \omega_{33j}x_j'-\omega_{13j} \\
            \omega_{31j}y_j'-\omega_{21j} &
            \omega_{32j}y_j'-\omega_{22j} &
            \omega_{33j}y_j'-\omega_{23j} \\
          \end{bmatrix}
          \begin{bmatrix}
            u \\ v \\ w
          \end{bmatrix}
          &=
          \begin{bmatrix}
            \tau_{xj}-\tau_{zj}x_j' \\
            \tau_{yj}-\tau_{zj}y_j' \\
          \end{bmatrix}
        \end{align*}
      }
    \item With multiple cameras, this system of equations can be expanded
  \end{itemize}
\end{frame}

\begin{frame}
  \frametitle{Discussion}
  \begin{itemize}[<+->]
    \item Again, we use the solution as the initial estimate for a nonlinear
      optimization procedure
    \item The method requires the knowledge of corresponding image points
      $\mathbf{x}_j$
    \item Extrinsic and intrinsic parameters must also be known
  \end{itemize}
\end{frame}

\section{Applications}

\begin{frame}
  \frametitle{Depth from structured light (Scharstein \& Szeliski, 2003)}
  \begin{columns}[c]
    \column{.5\textwidth}
  \includegraphics[width=1.0\textwidth]{fig/structured.png}
    \column{.5\textwidth}
  \includegraphics[width=1.0\textwidth]{fig/structured2.png}
  \end{columns}
\begin{itemize}[<+->]
  \item Projector geometry is modeled by the same equation as the camera
  \item Each image point receives a unique sequence of black and white values
  \item Use 3D world point inference to construct a 3D model
\end{itemize}

\end{frame}

\begin{frame}
  \frametitle{ProFORMA (Qi Pan, 2009)}
  \begin{itemize}
    \item \url{https://www.youtube.com/watch?v=vEOmzjImsVc}
    \item Does many other things too (feature detection, tracking,
      correspondence, model building...)
  \end{itemize}
\end{frame}

\end{document}
